%\setchapterimage{Header_Peugeot.jpg}
\setchapterpreamble[u]{\margintoc}
%\setcounter{chapter}{1}

\chapter{Modélisation cinématique}

\marginnote[8cm]{
%\UPSTIcompetence[2]{B2-14}
%\UPSTIcompetence[2]{C2-03}
}

%\begin{marginfigure}
%\includegraphics[width=4cm]{newton}%images/prot_01
% \capture{Isaac Newton -- 1643 - 1727.}
% \end{marginfigure}
% 


\begin{defi} [Solide Indéformable]
On considère deux points $A$ et $B$ d'un solide indéformable noté $S$. On note $t$ le temps. $
\forall A,B \in S, \forall t\in \mathbb{R}, \vect{AB(t)}^2 = \text{constante}
$.

\end{defi}

\begin{defi} [Trajectoire d'un point appartenant à un solide]

Soit un point $P$ se déplaçant dans un repère $\mathcal{R}_0 \left(O,\vect{i_0},\vect{j_0},\vect{k_0} \right)$. La trajectoire du point $P$ est définie par la courbe $\mathcal{C}(t)$ paramétrée par le temps $t$. On a : 
$$
\forall t\in \mathbb{R}^+, \vect{OP(t)}=
\left[
\begin{array}{l}
x(t)\\
y(t)\\
z(t)\\
\end{array}
\right]_{\mathcal{R}_0}
=x(t)\vect{i_0}+y(t)\vect{j_0}+z(t)\vect{k_0}
$$
\end{defi}



\begin{defi} [Vitesse d'un point appartenant à un solide]

Soit un solide $S_0$ auquel on associe le repère $\mathcal{R}_0$ $\left(O_0,\vect{i_0},\vect{j_0},\vect{k_0} \right)$.  Soit un solide $S_1$ auquel on associe le repère $\mathcal{R}_1$,  $\left(O_1,\vect{i_1},\vect{j_1},\vect{k_1} \right)$. Le solide $S_1$ est en mouvement par rapport au solide $S_0$. 
Soit un point $P$ appartenant au solide $S_1$. La vitesse du point $P$ appartenant au solide $S_1$ par rapport au solide $S_0$ se calcule donc ainsi : $
\vect{V(P\in S_1/S_0)}(t) = \left[\dfrac{\text{d}\vect{O_0P(t)}}{\text{d}t}\right]_{\mathcal{R}_0}.
$
\end{defi}

%\begin{exemple}
%
%\end{exemple}


\begin{resultat}
Lorsque il n'y a pas de degré de liberté de translation dans une liaison, la vitesse au centre de la liaison est nulle. Ainsi : 
\begin{itemize}
\item si les solides $S_1$ et $S_2$ sont en liaison rotule de centre $O$ alors $\vectv{O}{S_2}{S_1}=\vect{0}$;
\item si les solides $S_1$ et $S_2$ sont en liaison pivot de d'axe $\left(O,\vect{u}\right)$ alors $\vectv{O}{S_2}{S_1}=\vect{0}$;
\item si les solides $S_1$ et $S_2$ sont en liaison rotule à doigt de centre $O$ alors $\vectv{O}{S_2}{S_1}=\vect{0}$.
\end{itemize}
\end{resultat}


\begin{resultat}[Dérivation vectorielle]

Soient $S_0$ et $S_1$ deux solides en mouvements relatifs et $\mathcal{R}_0$ et $\mathcal{R}_1$ les repères orthonormés directs associés. Soit $\vect{v}$ un vecteur de l'espace. On note $\vect{\Omega(\mathcal{R}_1/\mathcal{R}_0)}$ le vecteur instantané de rotation permettant d'exprimer les rotations entre chacune des deux bases. La dérivée d'un vecteur dans une base mobile se calcule donc ainsi :

$$
\left[\dfrac{\text{d}\vect{v}}{\text{d}t}\right]_{\mathcal{R}_0} =
\left[\dfrac{\text{d}\vect{v}}{\text{d}t}\right]_{\mathcal{R}_1} 
+ \vect{\Omega(\mathcal{R}_1/\mathcal{R}_0)}\wedge \vect{v}.
$$
\end{resultat}


\begin{resultat}[Champ du vecteur vitesse dans un solide -- Formule de Varignon -- Formule de BABAR]

Soient $A$ et $B$ deux points appartenant à un solide $S_1$ en mouvement par rapport à $S_0$. Le champ des vecteurs vitesses est donc déterminé ainsi :
$$
\vect{V(\mathbf{B}\in S_1/S_0)} = \vect{V(\mathbf{A}\in S_1/S_0)} + \vect{\mathbf{BA}}\wedge \underbrace{\vect{\Omega(S_1/S_0)}}_{\mathbf{\vect{R}}}
$$

\end{resultat}


\begin{resultat}[Composition du vecteur vitesse]
\label{ref_va}

Soit un solide $S_1$ en mouvement par rapport à un repère $\mathcal{R}_0$ et un solide $S_2$ par rapport au solide $S_1$. Pour chacun des points $A$ appartenant au solide $S_2$, on a :
$$
\vectv{A}{S_2}{\mathcal{R}_0}=
\vectv{A}{S_2}{S_1}+\vectv{A}{S_1}{\mathcal{R}_0}
$$
\end{resultat}


\begin{remarque}
\begin{itemize}
\item $\vectv{A}{S_2}{\mathcal{R}_0}$ est appelé vecteur vitesse absolu;
\item $\vectv{A}{S_2}{S_1}$ est appelé vecteur vitesse relatif; 
\item $\vectv{A}{S_1}{\mathcal{R}_0}$ est appelé vecteur vitesse d'entraînement.
\end{itemize}
\end{remarque}

\begin{resultat}[Composition du vecteur vitesse]

Soit un solide $S_1$ en mouvement par rapport à un repère $\mathcal{R}_0$ et un solide $S_2$ par rapport au solide $S_1$. On a : 
$$
\vecto{S_2}{\mathcal{R}_0}=
\vecto{S_2}{S_1}+\vecto{S_1}{\mathcal{R}_0}
$$
\end{resultat}

\begin{defi}[Accélération d'un point appartenant à un solide]

Soit un solide $S_0$ auquel on associe le repère $\mathcal{R}_0$ $\left(O_0,\vect{i_0},\vect{j_0},\vect{k_0} \right)$.  Soit un solide $S_1$ auquel on associe le repère $\mathcal{R}_1$,  $\left(O_1,\vect{i_1},\vect{j_1},\vect{k_1} \right)$. Le solide $S_1$ est en mouvement par rapport au solide $S_0$. 


Soit un point $P$ appartenant au solide $S_1$. L'accélération du point $P$ appartenant au solide $S_1$ par rapport au solide $S_0$ se calcule donc ainsi : 
$$
\vect{\Gamma(P\in S_1/S_0)}(t) = \left[\dfrac{\dd \left( \vect{V(P\in S_1/S_0)}(t)\right)}{\dd t}\right]_{\mathcal{R}_0}
$$

\end{defi}

